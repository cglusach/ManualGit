\documentclass[10pt,landscape]{article}
\usepackage{multicol}
\usepackage{calc}
\usepackage{ifthen}
\usepackage[landscape]{geometry}
\usepackage{amsmath,amsthm,amsfonts,amssymb}
\usepackage{color,graphicx}
\usepackage{hyperref}
\usepackage[utf8]{inputenc}

\pdfinfo{
  /Title (Cheatsheet Comandos de Git)
  /Creator (TeX)
  /Producer (pdfTeX 1.40.0)
  /Author (Daniel Gacitúa Vásquez)
  /Subject (Comandos Git)
  /Keywords (cheatsheet, git)}

% This sets page margins to .5 inch if using letter paper, and to 1cm
% if using A4 paper. (This probably isn't strictly necessary.)
% If using another size paper, use default 1cm margins.
\ifthenelse{\lengthtest { \paperwidth = 11in}}
    { \geometry{top=.5in,left=.5in,right=.5in,bottom=.5in} }
    {\ifthenelse{ \lengthtest{ \paperwidth = 297mm}}
        {\geometry{top=1cm,left=1cm,right=1cm,bottom=1cm} }
        {\geometry{top=1cm,left=1cm,right=1cm,bottom=1cm} }
    }

% Turn off header and footer
\pagestyle{empty}

% Redefine section commands to use less space
\makeatletter
\renewcommand{\section}{\@startsection{section}{1}{0mm}%
                                {-1ex plus -.5ex minus -.2ex}%
                                {0.5ex plus .2ex}%x
                                {\normalfont\large\bfseries}}
\renewcommand{\subsection}{\@startsection{subsection}{2}{0mm}%
                                {-1explus -.5ex minus -.2ex}%
                                {0.5ex plus .2ex}%
                                {\normalfont\normalsize\bfseries}}
\renewcommand{\subsubsection}{\@startsection{subsubsection}{3}{0mm}%
                                {-1ex plus -.5ex minus -.2ex}%
                                {1ex plus .2ex}%
                                {\normalfont\small\bfseries}}
\makeatother

% Define BibTeX command
\def\BibTeX{{\rm B\kern-.05em{\sc i\kern-.025em b}\kern-.08em
    T\kern-.1667em\lower.7ex\hbox{E}\kern-.125emX}}

% Don't print section numbers
\setcounter{secnumdepth}{0}


\setlength{\parindent}{0pt}
\setlength{\parskip}{0pt plus 0.5ex}

%My Environments
\newtheorem{example}[section]{Example}


%% ------------------------------------------------------------------------- %%

\newcommand{\tablainicio}{
    \begin{tabular}{@{}ll@{}}
}

\newcommand{\tablafin}{
    \end{tabular}
}

\newcommand{\tablacomando}[2]{
    \texttt{#1} & #2 \\
}

\newcommand{\comandolargo}[2]{
    \raggedright{\small{\texttt{#1}}} \\
    \raggedleft{\footnotesize{#2}} \\
    \raggedright
}

\newcommand{\comandomultiple}[1]{
    {\footnotesize
    {\ttfamily
        #1
    }
    }
}

\newcommand{\imagen}[2]{
    \begin{center}
    \noindent \includegraphics[#1]{#2}
    \end{center}
}

\begin{document}

\LARGE{Cheatsheet: Comandos Esenciales de Git}\\
\small{Elaborado por \textbf{Daniel Gacitúa Vásquez} para la \textbf{Comunidad GNU/Linux USACH}}\\
\small{Más documentación en \textsf{http://cglusach.github.io/manualgit/}}

\raggedright
\raggedcolumns
%%\footnotesize
\begin{multicols}{3}


% multicol parameters
% These lengths are set only within the two main columns
%\setlength{\columnseprule}{0.25pt}
\setlength{\premulticols}{1pt}
\setlength{\postmulticols}{1pt}
\setlength{\multicolsep}{1pt}
\setlength{\columnsep}{2pt}

\section{Instalar Git en GNU/Linux}

\subsection{Instalar Git:}

\comandomultiple{sudo apt-get install git kdiff3-qt}
    
\subsection{Configurar Git:}
    
\comandomultiple{
    git config --global user.name "[NOMBRE] [APELLIDO]"\\
    git config --global user.email [CORREO]\\
    git config --global merge.tool kdiff3\\
}
    
\subsubsection{Crear llave SSH:}

\comandomultiple{
    ssh-keygen -t rsa -b 4096 -C "[CORREO]"\\
    eval "\$(ssh-agent -s)"\\
    ssh-add ~/.ssh/id\_rsa\\
    cat ~/.ssh/id\_rsa.pub\\
}
    
\footnotesize Copiar la salida del comando \texttt{cat} en el gestor de repositorios remotos (GitHub, BitBucket, etc.)


\section{Iniciar un repositorio}

\comandolargo{git init}{Inicia un Repositorio Git vacío}
\comandolargo{git remote add [REMOTO] [URL]}{Establece una URL de Repo Remoto (usar con \texttt{git init})}
\comandolargo{git clone [URL]}{Copia un Repositorio Remoto desde la URL indicada}


\section{Comandos Básicos}

\comandolargo{git add [ARCHIVO]}{Agregar un archivo al Index}
\comandolargo{git add --all}{Agregar todos los archivos al Index}
\comandolargo{git commit -m "[MENSAJE]"}{Generar un commit con los archivos en el Index}
\comandolargo{git pull [REMOTO] [RAMA]}{Tirar commits desde el Repositorio Remoto}
\comandolargo{git push [REMOTO] [RAMA]}{Empujar commits hacia el Repositorio Remoto}


\section{Estado actual del repositorio}

\comandolargo{git status}{Revisar estado del Repositorio Local}
\comandolargo{git log}{Revisar últimos commits en el Repositorio Local}
\comandolargo{git diff}{Muestra las diferencias entre el Directorio Local y el Index}




\columnbreak

\section{Merging}

\comandolargo{git mergetool}{Abrir herramienta de fusionado}


\section{Rollbacking}

\comandolargo{git revert [HASH]}{Revertir commit, generando uno nuevo}
\comandolargo{git revert --no-commit HEAD\~{}\#..HEAD}{Revertir últimos `\texttt{\#}' commits, dejando los cambios en el Index}
\comandolargo{git reset [HASH]}{Deshacer permanentemente los cambios hasta el commit}
\comandolargo{git reset --hard HEAD}{Deshacer cambios locales hasta el último commit}


\section{Tagging}

\comandolargo{git tag -a [VERSIÓN] -m "[MENSAJE]"}{Etiqueta el último commit}
\comandolargo{git tag -a [VERSIÓN] -m "[MENSAJE]" [HASH]}{Etiqueta el commit indicado}
\comandolargo{git push [REMOTO] --tags}{Empujar todas las etiquetas al Repositorio Remoto}
\comandolargo{git tag}{Ver lista de etiquetas del proyecto}
\comandolargo{git show [VERSIÓN]}{Ver la información del commit asociada al tag}


\section{Stashing}

\comandolargo{git stash}{Crea un stash y limpia el Directorio Local}
\comandolargo{git stash list}{Muestra una lista de los stashes en el Repositorio Local}
\comandolargo{git stash apply}{Aplica el último stash sobre el Repositorio Local}
\comandolargo{git stash apply stash@\{\#\}}{Aplica el stash numerado sobre el Repo Local (ej: \texttt{stash@\{1\}})}
\comandolargo{git stash drop stash@\{\#\}}{Elimina el stash numerado del Repo Local (ej: \texttt{stash@\{1\}})}


\section{Branching}

\comandolargo{git branch [RAMA]}{Crear una nueva rama}
\comandolargo{git checkout [RAMA]}{Cambiar a la rama indicada}
\comandolargo{git merge [RAMA]}{Fusionar cambios desde la rama indicada a la actual}


\columnbreak

\section{Ejemplos}

\subsection{Comandos Básicos}

\comandomultiple{
    git init\\
    git remote add origin git@github.com:user/proyecto.git\\
    git add README.txt\\
    git add main.c\\
    git commit -m "Añadidos README y main.c"\\
    git push origin master\\
    git pull origin master
}

\subsection{Otros Comandos}

\comandomultiple{
    git revert 5e05d51\\
    git revert --no-commit HEAD\~{}10..HEAD\\
    git reset 6f11a20\\
    git tag -a v1.0 -m "Versión final del producto"\\
    git tag -a v1.1 -m "Hotfix de día cero" f12a56c\\
    git push origin --tags\\
    git show v1.1\\
    git branch fix01\\
    git checkout master\\
    git merge fix01
}

\section{Zonas de trabajo de Git}

{\footnotesize

\begin{itemize}
\item Workspace (o Directorio Local)
\item Index (o Índice)
\item Local Repository (o Repositorio Local)
\item Remote Repository (o Repositorio Remoto)
\end{itemize}

}

\imagen{width=7.5cm}{imagenes/gitworkflow.png}

\end{multicols}

\end{document}